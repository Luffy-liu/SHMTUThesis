% degree=[master]                             % 必选,学位类型
% language=[chinese|english],                 % 可选(默认:chinese),论文的主要语言
% review=[true|false]					      % 可选,盲审,默认为false
% bibstyle=[gb7714-2015|gb7714-2015ay|ieee],  % 可选(默认:gb7714-2015),参考文献样式
\documentclass[degree=master, language=chinese, review=false, oneside, AutoFakeBold]{shmtuthesis}

% 所有其它可能用到的包都统一放到shmtuthesis.sty中,可以根据自己的实际添加或者删除。
\usepackage{shmtuthesis}

% 定义图片文件目录与扩展名
\graphicspath{{figures/}}
\DeclareGraphicsExtensions{.pdf, .eps, .png, .jpg, .jpeg}

% 导入参考文献数据库
\addbibresource{thesis.bib}

% 论文信息,必须
\input{tex/information}

\begin{document}
	
	% 无编号内容:论文封面、授权页
	\maketitle
	\makeDeclareOriginality
	\makeDeclareAuthorization
	
	% 使用罗马数字对前言编号
	\frontmatter
	
	% 摘要
	\input{tex/abstract}
	
	% 目录、插图目录、表格目录、算法目录
	\tableofcontents
	\listoffigures
	\listoftables
	\listofalgorithms
	
	% 使用阿拉伯数字对正文编号
	\mainmatter
	
	% 正文内容
	\input{tex/introduction}
	\input{tex/float}
	\input{tex/math_and_citations}
	\input{tex/summary}
	
	% 致谢
	\input{tex/acknowledgements}
	
	% 参考文献
	\printbibliography[heading=bibintoc]
	
	% 使用英文字母对附录编号
	\appendix
	\input{tex/appendix/maxwell_equations}
	\input{tex/appendix/flow_chart}
	
	% 文后无编号部分
	\backmatter
	
	% 发表论文、获奖情况、申请专利、参与项目
	% 盲审论文中,发表学术论文及参与科研情况等仅以第几作者注明即可,不要出现作者或他人姓名
	\input{tex/achievements/publications}
	\input{tex/achievements/awards}
	\input{tex/achievements/patents}
	\input{tex/achievements/projects}
	
\end{document}